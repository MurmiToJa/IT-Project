\documentclass{article}
\usepackage{graphicx} 
\usepackage{enumitem}
\usepackage{geometry}
\usepackage{tocloft}
\usepackage{fancyhdr}
\usepackage{hyperref}

\renewcommand{\contentsname}{Table Of Content}
\renewcommand{\cftsecfont}{}
\renewcommand{\cftsubsecfont}{}
\renewcommand{\cftsubsubsecfont}{}

\pagestyle{fancy}
\fancyhf{} 
\fancyhead[C]{FinTax} 
\fancyhead[L]{}
\fancyhead[R]{}
\fancyfoot[C]{\thepage} 



\title{\textbf{SILESIAN UNIVERSITY OF TECHNOLOGY FACULTY OF AUTOMATIC CONTROL, ELECTRONICS AND COMPUTER SCIENCE - FinTax - Raport 2.
}}
\author{Kacper Szymaniak Bartłomiej Murmyłowski \\
Group: 5TI Section: 533
}

\date{26 October 2023}


\begin{document}
\thispagestyle{empty}
\newgeometry{left=5cm, right=5cm, top=3cm, bottom=3cm}   
    \maketitle
 \restoregeometry
    \newpage
\newgeometry{left=2.5cm, right=2.5cm, top=3cm, bottom=3cm}   
\fontsize{14}{16}\selectfont

\tableofcontents

    \newpage
    \section{Introduction}
This week, our primary goal was to connect the domain and hosting, while our secondary objectives included familiarizing ourselves with the fundamentals of creating a proper HTML structure and designing the website layout in Figma.
    \section{Domain & Hosting}

Our client has acquired one of Home.pl's hosting packages, in addition to securing the domain FinTax.pl. Our primary objective for this week was to establish the connection between the purchased domain and the hosting package.

To facilitate this process, our client thoughtfully created hosting accounts for us, granting access for domain hosting and the preliminary setup required for the website. Unfortunately, a technical hurdle emerged that hindered our access to the hosting account. Given that our client is currently on a business trip, we find it necessary to postpone this task until a later date. Nevertheless, we remain fully committed to resolving this issue as promptly as possible.

\section{Layout}
To avoid wasting time, Kacper has turned his attention to another aspect of the visual design of the website. He has begun creating a visualization of the site using the Figma application, drawing upon his previous report's work as a foundation. Initially, we had planned to build upon what we created last week, but we have come to the conclusion that, since we have plenty of time to spare at the moment, we will use it to make our future work more manageable.

As Kacper is still working on this, we are presenting a portion of the work here, and the rest will be uploaded to our \href{https://github.com/MurmiToJa/IT-Project}{GitHub repository}  under raport number 3.

\begin{figure}[h]
    \centering
    \includegraphics[width=1\linewidth]{fintax.png}
    \caption{Home - first outline..}
    \label{fig:enter-label}
\end{figure}


\section{Introduction To HTML}   
This week, we also began to familiarize ourselves with HTML structures. We learned or refreshed our understanding of the basic hierarchy, attributes, semantics, and proper validation.

Additionally, we set up our working environment in Visual Studio Code to make it as user-friendly as possible:

We installed themes and plugins that suit our needs, including HTML CSS support.
Our learning resources include:


\begin{itemize}
    \item https://www.w3schools.com/
    \item https://web.dev/learn/html
    \item https://developer.mozilla.org/
\end{itemize}

Our progress can be tracked on our \href{https://github.com/MurmiToJa/IT-Project}{GitHub repository.} 


\end{document}


